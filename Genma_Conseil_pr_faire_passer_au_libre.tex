\documentclass{beamer}
\mode<presentation> {
%\usetheme{Madrid}
%\usetheme{default}
\usepackage{color}
\definecolor{bottomcolour}{rgb}{0.21,0.11,0.21}
\definecolor{middlecolour}{rgb}{0.21,0.11,0.21}
\setbeamercolor{structure}{fg=white}
\setbeamertemplate{frametitle}[default]%[center]
\setbeamercolor{normal text}{bg=black, fg=white}
\setbeamertemplate{background canvas}[vertical shading]
[bottom=bottomcolour, middle=middlecolour, top=black]
\setbeamertemplate{items}[circle]
\setbeamertemplate{navigation symbols}{} %no nav symbols
\setbeamercolor{block title}{use=structure,fg=white,bg=structure.fg!50!red!50!blue!100!green}
\setbeamercolor{block body}{parent=normal text,use=block title,bg=block title.bg!5!white!10!bg,fg=white}
\setbeamertemplate{navigation symbols}{}
\newcounter{moncompteur}
}

\usepackage{graphicx} 
\usepackage{booktabs} 
\usepackage[utf8]{inputenc}  
\usepackage[T1]{fontenc}  
\usepackage{geometry}     
\usepackage[francais]{babel} 
\usepackage{eurosym}
\usepackage{verbatim}
\usepackage{ragged2e}
\justifying

\input{cc_beamer}

\title[Conseils aux libristes pour faire passer au libre]{Conseils aux libristes pour faire passer au libre} 
\author{Genma}

\begin{document}

%% Titlepage
\begin{frame}
	\titlepage
	\vfill
	\begin{center}
		\CcGroupByNcSa{0.83}{0.95ex}\\[2.5ex]
		{\tiny\CcNote{\CcLongnameByNcSa}}
		\vspace*{-2.5ex}
	\end{center}
\end{frame}

%----------------------------------------------------------------------------------------
\begin{frame}
\justifying{
Cette présentation est une adaptation /s'inspire du retour d'expérience "Conseils à un libriste pour faire passer au libre" de Dada, qui est un texte mis à disposition selon les termes de la Licence Creative Commons Attribution - Partage dans les Mêmes Conditions 4.0 International. \url{http://www.dadall.info/blog}
}
\end{frame}

%------------------------------------------------
\begin{frame}
\frametitle{Conseil n\degre \themoncompteur }

\begin{block}{Ne pas forcer le passage}
\justifying{
\begin{itemize}
\item Il ne sert à rien de se lever un jour en disant qu'untel va avoir le droit à son passage au libre. 
\item Il s'en fout, n'a pas connaissance des courants privateur et libre des logiciels. 
\item L'utilisateur d'un ordinateur veut que ça marche et n'aime pas le changement.
\end{itemize}
}
\end{block}
\end{frame}
\addtocounter{moncompteur}{1}

%------------------------------------------------
\begin{frame}
\frametitle{Conseil n\degre \themoncompteur }
\begin{block}{Attendez que l'ordinateur se dégrade avant d'agir.}
\justifying{
\begin{itemize}
\item Prendre l'ordinateur d'un ami et lui installer quelque chose qu'il ne connait pas parce que vous vous trouvez ça bien, c'est trop souvent foncer dans le mur. 
\item Quand ça marche, ça marche et il ne veut surtout pas que ça change. 
\item Attendez qu'il veuille changer, qu'on lui répare l'ordinateur...
\end{itemize}
}
\end{block}
\end{frame}
\addtocounter{moncompteur}{1}

%------------------------------------------------
\begin{frame}
\frametitle{Conseil n\degre \themoncompteur }
\begin{block}{Commencer simplement, par des logiciels courants}
\justifying{
\begin{itemize}
\item Firefox, Thunderbird et VLC sont les meilleurs moyens au monde pour faire glisser quelqu'un vers les logiciels libres.
\item  Ils regroupent les besoins de 90 \% des utilisateurs.
\end{itemize}
}
\end{block}
\end{frame}
\addtocounter{moncompteur}{1}

%------------------------------------------------
\begin{frame}
\frametitle{Conseil n\degre \themoncompteur }
\begin{block}{Ne parlez pas de logiciel libre tout de suite}
\justifying{
\begin{itemize}
\item L'ordinateur est une boite noire. 
\item Personne n'y comprend rien et personne ne veut comprendre. 
\item Alors laissez tomber l'approche éthique de votre démarche si vous savez que la personne en face n'est pas sensible à ça. 
\end{itemize}
}
\end{block}
\end{frame}
\addtocounter{moncompteur}{1}

%------------------------------------------------
\begin{frame}
\frametitle{Conseil n\degre \themoncompteur }
\begin{block}{GNU/Linux, privateur, RMS...}
\justifying{
\begin{itemize}
\item Ne commencez à dire que Linux, c'est bien mieux que Windows.
\item Ne parlez pas de la philosophie, des 4 libertés, de RMS...
\item Parlez d'Ubuntu, dont le nom peut être connu.
\item ET NE PARLEZ PAS DE GNU/Linux.
\end{itemize}
}
\end{block}
\end{frame}
\addtocounter{moncompteur}{1}


%------------------------------------------------
\begin{frame}
\frametitle{Conseil n\degre \themoncompteur }
\begin{block}{Ne commencez jamais par le système d'exploitation}
\justifying{
\begin{itemize}
\item Non parlez pas du fonctionnement du système d'exploitation. 
\item ne parlez pas de la ligne de commande qui est trop bien.
\end{itemize}
}
\end{block}
\end{frame}
\addtocounter{moncompteur}{1}

%------------------------------------------------
\begin{frame}
\frametitle{Conseil n\degre \themoncompteur}
\begin{block}{Refusez d'aider un utilisateur de logiciels piratés}
\justifying{
\begin{itemize}
\item Installer lui du logiciel libre.
 \item Non Gimp ce n'est pas Photoshop, il faut REAPPRENDRE, se créer de nouvelles habitudes.
\end{itemize}
}
\end{block}
\end{frame}
\addtocounter{moncompteur}{1}

%------------------------------------------------
\begin{frame}
\frametitle{Conseil n\degre \themoncompteur }
\begin{block}{Assumez le Service Après Vente}
\justifying{
\begin{itemize}
\item Assumez le SAV. 
\item N'oubliez pas de configurer les mis à jour des versions d'Ubuntu vers les LTS suivantes uniquement.
\end{itemize}
}
\end{block}
\end{frame}
\addtocounter{moncompteur}{1}

%------------------------------------------------
\begin{frame}
\frametitle{Conseil n\degre \themoncompteur }
\begin{block}{Laissez tomber si c'est un joueur}
\justifying{
N'y pensez même pas, c'est tout.
\\
Et oui je connais STEAM OS.
}
\end{block}
\end{frame}
\addtocounter{moncompteur}{1}

%----------------------------------------------------------------------------------------
\begin{frame}
\Huge{\centerline{Merci de votre attention.}}
\Huge{\centerline{Place aux questions. Débattons...}}
\end{frame}

\end{document}
